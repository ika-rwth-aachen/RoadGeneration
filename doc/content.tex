%!TEX root = ../main.tex
\thispagestyle{ika}

% --- EINLEITUNG ---------------------------------------------------------------
\chapter{Einleitung}

In der Automobilindustrie geht der Trend in den vergangenen Jahren verstärkt  zur Entwicklung und Realisierung von autonomen Fahrfunktionen. Um diese automatisierten Fahrmanöver zu validieren und abzusichern werden Testfahrten mit oft vielen tausend Testkilometern benötigt. Als kostengünstige Alternative zur realen Testfahrt können diese Fahrmanöver auch mithilfe von Simulationen durchgeführt werden. Der Ausgangspunkt bzw. die Grundvoraussetzung für jede Fahrzeugsimulation ist eine vorhandene Streckenbeschreibung auf der die entsprechende Simulation durchgeführt wird. Zum einen können solche Strecken anhand von realen Daten erzeugt werden, was jedoch einen hohen Detaillierungsgrad voraus setzt und aktuell nicht automatisiert realisierbar ist. Eine weitere Möglichkeit zur Erzeugung von Strecken ist ein logisch, variierbarer Ansatz der als Grundlage für diese Seminararbeit herangezogen wird. Hierbei ist nicht das Abbild der Realität das Ziel, sondern Vorschriften und Grenzen einzuhalten und durch einfache und parametrisierte Angaben ein komplexes und vor allem in sich valides Streckennetz generieren zu können.

Die Grundidee für diese Arbeit ist die Implementierung eines Tools mithilfe dessen der Benutzer ein vereinfachtes Eingabeformat füllt und als Resultat ein komplexes, aber in sich valides Streckennetz erhält. Das Eingabeformat wurde als solches im Rahmen einer Masterarbeit klar definiert und ausgearbeitet \cite{Russ.2019}. Die Hauptaufgabe dieser Seminararbeit besteht nun also in der Validierung und Absicherung des vorhandenen Formates und dem Entwickeln eines Übersetzungstool, dass die abstrakte Strecke in einem industriell gängigem Format als Resultat liefert.

% --- ALLGEMEINE STRECKENBESCHREIBUNG ------------------------------------------
\chapter{Allgemeine Streckenbeschreibung}

Grundsätzlich lässt sich eine Strecke in verschiedene Ebenen teilen, wie in Abbildung \ref{abb1} sichtbar. Ausgangspunkt ist die Straße als solche mit Fahrbahnen und Fahrbahnmarkierungen. Ein weiterer Bestandteil sind verkehrsführende Elemente wie beispielsweise Beschilderungen, Ampelanlagen oder Bushaltestellen. Ein letzter, für die Streckenbeschreibung notwendiger Aspekt, sind temporäre Veränderungen wie zum Beispiel Baustellen. Alle weiteren Ebenen des gezeigten Modells finden in der Streckenbeschreibung und Generierung keine Anwendung.

\begin{figure}[H]
\flushleft
\includegraphics[width=0.95\textwidth]{fig/fig1.png}
\caption{Vereinfachtes Ebenenmodell einer Straße \cite{Eckstein.2018}}
\label{abb1}
\end{figure}

Im Allgemeinen gibt es zwei relevante Koordinatensysteme für Strecken. Neben dem globalen x-y Koordinatensystem existiert das der Referenzlinie mitbewegte s-t Koordinatensystem, wie auch Abbildung \ref{abb2} verdeutlicht. Jede Straße besitzt eine Referenzlinie, deren Beschreibung analytisch durch das Verbinden von Gerade, Kreisbögen und Spiralsegmenten möglich ist. Diese im Straßenbau standardisierte Methode ermöglicht eine kontinuierliche Beschreibung ohne Krümmungsprung. Eine Gerade mit Krümmung Null und ein Kreisbogen mit konstanter Krümmung können durch ein Spiralsegment mit linearem Krümmungsanstieg verbunden werden. Eine beispielhafte Referenzlinie ist hierbei in Abbild \ref{abb2} dargestellt, wobei das unten dargestellte Krümmungsband wie erläutert stetig ist.

\begin{figure}[H]
\flushleft
\includegraphics[width=0.95\textwidth]{fig/fig2.png}
\caption{Beispielhafte Referenzlinie im s-t-z Koordinatensystem \cite{Becker.2017}}
\label{abb2}
\end{figure}

An die definierte Referenzlinie können dann im Anschluss Fahrbahnen hinzugefügt werden, für welche individuelle Breiten oder Markierungen festgelegt werden können. Die Verbindung von mehreren auf diese Weise definierten Strassen ermöglicht dann die Darstellung komplexer Strekennetze und gewährleistet somit eine eindeutige Streckenbeschreibung.

% --- OPEN DRIVE STANDARD ------------------------------------------------------
\chapter{OpenDRIVE Standard}

OpenDRIVE ist ein 2005 entwickeltes Format zur Darstellung von logischen Straßennetzwerken und gilt mittlerweile als Standard für Streckenbeschreibungen in Fahrzeugsimulationen. Das Format enthält nicht nur Straßen und Fahrbahninformationen, sondern ermöglicht auch die Definition komplexerer Beschilderungen, Fahrbahnmarkierungen oder Ampelsignale. \cite{OpenDRIVE.2019} Definiert wird die OpenDRIVE Datenstruktur im XML-Format in welcher grundsätzlich zwischen Straßen (\texttt{roads}) und Verbindungsstücken (\texttt{junctions}) unterschieden wird. Eine vereinfachte Übersicht der OpenDRIVE Struktur lässt sich in Abbildung\ref{abb3} finden. Hieraus wird deutlich, dass Straßen sowohl direkt, als auch über \texttt{junctions} miteinander verbunden werden können. 

Die Referenzlinie einer jeden Straße wird wie erläutert mit geometrischen Elementen analytisch beschrieben, wobei zusätzlich die laterale Position und Ausrichtung, sowie das Höhenprofil der Straße definiert werden können. Hinzugefügt werden der Straße dann einzelne Fahrbahnen (\texttt{lanes}). Zudem lassen sich unbegrenzt Objekte wie Ampeln oder Schildern definieren. \cite{OpenDRIVEDoku.2019} Das openDRIVE Format ist somit beliebig erweiterbar und ermöglicht dadurch die eindeutige Beschreibung jeder möglichen Strecke. Durch die hohe Komplexität des Formats wird dieses bei großen Streckennetzen für einen Nutzer aber kaum von Hand generierbar.

\begin{figure}[H]
\flushleft
\includegraphics[width=0.95\textwidth]{fig/fig3.png}
\caption{Vereinfachte OpenDRIVE Struktur \cite{Dupuis.2006}}
\label{abb3}
\end{figure}

Deshalb existieren Tools mit denen ermöglicht wird eine Strecke über eine grafische Benutzeroberfläche im OpenDRIVE Format zu erzeugen. Ein Beispiel hierfür ist der \texttt{Road Generator}.\cite{RoadGenerator.2019} Zur Abbildung der Realität kann dies herangezogen werden, allerdings ist die generische Variierbarkeit von Strecken durch einfache Parameteränderungen ein großer Vorteil bei der Simulation von Fahrmanövern die zur Absicherung durchgeführt werden. 

% --- ANNAHMEN UND VEREINFACHUNGEN IM IKA KONZEPT ------------------------------
\chapter{Annahmen und Vereinfachungen im ika Konzept}

Um eine logisch variierbare Straße im OpenDRIVE Format zu erzeugen, bedarf es ein Tool, dass anhand eines vereinfachten, benutzerfreundlichem Eingabeformat das komplexe, detailreiche OpenDRIVE Format generiert. Das Konzept dieses Eingabeformats und die dazugehörigen Vereinfachungen werden in diesem Abschnitt erläutert.

Anders als im OpenDRIVE Format werden Kreuzungen vom Nutzer nur über die Ausgangsstraßen und einem gemeinsamen Kreuzungspunkt definiert, die Definition der Verbindungsstraßen im Kreuzungsbereich sind nicht notwendig. Der Nutzer spezifiziert lediglich Eigenschaften wie die miteinander zu verbindenden Fahrbahnen oder die Größe des Kreuzungsbereichs. Auch werden Spuraufweitungen für Linksabbiegerspuren auf Hauptstraßen oder Fahrbahnmarkierungen innerhalb des Kreuzungsbereiches automatisiert generiert. Die Vielfalt bei der Generierung von Kreuzungen ist dadurch zwar eingeschränkter, jedoch für die Simulationen von Fahrmanövern ausreichend. So können durch wenige Parameteränderungen wie beispielsweise dem Schnittwinkel unter welchem sich die zwei Straßen einer X-Kreuzung schneiden beliebig viele und valide Kreuzungen generiert werden. 

Zudem muss der Benutzer keine expliziten Koordinaten der Straßen angeben, da die Annahme getroffen wird, dass der Kreuzungsmittelpunkt im x-y Ursprung liegt. Darüber hinaus gibt es die Möglichkeit verschiedene Kreuzungssegmente über zwei klar definierte Straßenpunkte miteinander zu verbinden. Über die x-y Koordinaten und die Winkel der Straßen ist die Lage der beiden Segmente eindeutig bestimmt. Das Verbinden von mehreren Elementen ermöglicht so die Erstellung komplexer Strassennetze mit nur wenig Nutzereingaben.

Um ein Straßennetzwerk schließen zu können bedarf es einer sehr genauen Konstruktion des letzten Segmentes, welches sich in einem überbestimmten Problem widerspiegelt. Deshalb wird zum Schließen eines Netzwerks eine separate Funktion entwickelt, die zwei Punkte unter entsprechendem Winkel miteinander verbindet. Dabei wird angenommen, dass die genaue Geometrie des letzten Segments nicht von großer Bedeutung ist.

Für den Nutzer sind also nur vereinfachte Angaben wie der Kreuzungstyp, die Straßengeometrie, der Kreuzungsschnittwinkel oder die Fahrbahnanzahl gefragt. Das Tool trifft nun Annahmen und Vereinfachungen, sodass basierend auf diesen die Übersetzung in ein detailreiches Straßennetzwerk im gängigen OpenDRIVE Format möglich wird.

% --- UMSETZUNG UND IMPLEMENTIERUNG DES TOOLS ----------------------------------
\chapter{Umsetzung und Implementierung des Tools}

\section{Ein - und Ausgabeformat}
Die Definition im Eingabeformat geschieht ebenfalls wie die Darstellung des OpenDRIVE Formats mithilfe einer XML-Struktur. Dies erlaubt eine übersichtliche Definition der Datenstruktur mit XML Attributen und Elementen. Das Tool zur Generierung des OpenDRIVE Formats wird in \texttt{C++} geschrieben, sodass eine kompatible Bibliothek zum Einlesen und Schreiben der XML-Dateien verwendet wird. Die Bibliothek \texttt{pugi\_xml} ermöglicht das Einlesen der Eingabedatei und speichert die XML-Struktur ähnlich einer Baumstruktur zur Laufzeit, was entsprechend eine effiziente und übersichtliche Verwendung der Eingaben innerhalb des Tools ermöglicht. Ebenso wird eine solche XML-Stuktur zur Laufzeit des Tools erzeugt, die dann mit den generierten OpenDRIVE Objekten gefüllt werden kann.

Ein weiterer Vorteil bei der Benutzung von XML Datenstrukturen ist die Möglichkeit XML Schemata (XSD) zu nutzen. Eine XSD Datei beschreibt die Struktur eines XML Dokumentes, sodass das Auftreten von Attributen und Elementen eingeschränkt werden kann. So besteht beispielsweise die Möglichkeit die Anzahl von Elementen einzuschränken, die Anordnung von Elementen innerhalb des XML Dokuments zu bestimmen oder Attribute nur optional vorzuschreiben. So wurde im Rahmen der Seminararbeit ein XML-Schema Dokument zur Absicherung von falschen Benutzereingaben erstellt. Das verwendete OpenDRIVE Format besitzt ebenfalls eine solche Schemadatei, sodass sowohl Eingabe als auch Ausgabedokument des Streckengenerators mithilfe eines \texttt{XML/XSD}Validierungstools abgesichert werden können.

\section{Implementierung}
Die Implementierung des in \texttt{C++} entwickelten Tools gliedert sich ebenfalls in die drei schematisch beschriebenen Entwicklungsschritte. Im ersten Schritt werden die Segmente (Kreuzungen, Kreisverkehre oder Verbindungsstraßen) basierend auf den Angaben in der Eingabedatei generiert. Im Anschluss werden diese Segmente dann wie vom Benutzer definiert miteinander verbunden bevor im letzten Schritt Straßen zum Schließen des Streckennetzwerkes generiert werden.

\subsection{Generierung der Segmente}
Die Eingabedaten werden zur Laufzeit eingelesen und sind so global als Baumstruktur verfügbar. Im Folgenden wird der Entwicklungsprozess einer X-Kreuzung genauer beschrieben.

Für eine X-Kreuzung existieren drei verschiedene Fälle wie in Abbildung \ref{abb4} dargestellt, welche sich in Anzahl und Anordnung der definierten Eingangsstraßen unterscheiden.

\begin{figure}[H]
\flushleft
\includegraphics[width=0.95\textwidth]{fig/fig4.tikz}
\caption{Unterschiedliche Arten einer X-Kreuzung}
\label{abb4}
\end{figure}

Abhängig vom definierten Kreuzungstyp werden die entsprechend relevanten Straßen und die zugehörigen Informationen wie beispielsweise die Kreuzungsbereichgröße aus dem XML Dokument eingelesen. Im nächsten Schritt werden vier Ausgangsstraßen der X-Kreuzung basierend auf den Eingangsstraßen generiert, welche sich entsprechend dem definierten Winkel anordnen lassen. Die generische Funktion \texttt{generateRoad} erstellt dabei eine Straße ausgehend vom bestimmtem Hauptpunkt im Kreuzungsbereich. Die positive s Koordinatenrichtung zeigt dabei immer weg vom Kreuzungsmittelpunkt. Zudem werden für Hauptstrassen automatisch Linksabbiegerspuren oder Ampelanlagen eingefügt.

Nach erfolgreicher Erstellung der vier Ausgangsstraßen werden die Straßen im Kreuzungsbereich erstellt. Der Benutzer hat die Möglichkeit alle Fahrbahnen automatisch und logisch miteinander zu verbinden oder selbst Verbindungsstraßen zu spezifizieren. Beim Definieren einer Fahrbahn ist darauf zu achten, dass sich mögliche Unterschiede im T-Offset oder der Spurbreite kontinuierlich ausgleichen. Die Funktion \texttt{createRoadConnection} verbindet die Fahrbahnen miteinander und generiert eine Verbindungsstraße mit vorgegebener Fahrbahnmarkierung.

Im Kreuzungsbereich besitzt jede angelegte Straße als Vor und Nachfolger eine \texttt{junction}, welche ebenfalls innerhalb der Funktionen festgelegt und definiert werden. Die zugrundeliegende Struktur lässt sich nochmals in Abbildung \ref{abb3} erkennen.

Analog werden auch die weiteren Segmente wie beispielsweise eine T-Kreuzung, ein Kreisverkehr oder auch eine einfache Verbindungsstraße generiert und in einer global verfügbaren Datenstruktur abgespeichert.

\subsection{Anordnung der Segmente}

Durch den Benutzer wird in der Eingabedatei das Referenzsegment mit Position und Ausrichtung festgelegt. Alle Geometrien dieses Segments werden also innerhalb des Tools um den definierten Winkel gedreht und entlang der x-y Achse verschoben, sodass sich der Ursprung des Segments (Kreuzungsmittelpunkt) an definierter Position befindet.

Weiterhin gibt der Nutzer Straßenpunkte zweier unterschiedlicher Segmente an, die miteinander verbunden werden sollen. Innerhalb der Funktion \texttt{linkSegments} wird dann der Winkel um den gedreht werden muss und die x-y Werte der Verschiebung bestimmt, sodass nach Transformation der Geometrien des neuen Segments die Endpunkte der zwei Segemente ohne Winkelsprung aufeinander liegen.

\subsection{Verbinden der Segmente}

Im letzten Schritt werden automatisch Straßen generiert, die zwei beliebige Endpunkte zweier Segmente miteinander verbinden, was im Entwicklungsprozess zum Schließen eines Streckennetzwerkes erforderlich ist. Die verantwortliche Funktion \texttt{closeRoadConnection} berechnet hierbei aus den Koordinaten und Richtungen der Punkte die Verbindungsstrecke. Dazu wird der Schnittpunkt der beiden Richtungsvektoren gebildet und unterschieden, ob der Schnittpunkt auf positiver oder negativer Seite des Schnittpunktes liegt. Dies führt letztendlich zu neun verschiedenen Fällen, je nach Ausrichtung der beiden zu verbindenden Punkte. Falls eine direkte Verbindung mithilfe einer Geraden und einer Verbundkurve nicht möglich ist, wird ein Hilfspunkt eingefügt, die Strecke bis dahin mit Gerade oder Kreisbogen erweitert und entsprechend mit den neuen Punkten die Funktion \texttt{closeRoadConnection} rekursiv aufgerufen. Der Fall, dass die Punkte direkt miteinander verbunden werden können, lässt sich in zwei Unterfälle teilen.

Zum einen ist dies möglich, falls beide Vektoren in die gleiche Richtung und der Startvektor auf den Endvektor zeigt. Für diesen trivialen Fall lässt sich das Stück mit einer einfachen Geraden verbinden.

Für den Fall, dass der Schnittpunkt in positiver Richtung des Startvektors und in negativer Richtung des Endvektors liegt lässt sich die Strecke ebenfalls verbinden. Dazu wird der Abstand von Start und Endpunkt zum Schnittpunkt verglichen und das längere Stück so weit mit einer Geraden verlängert, bis Start und Endpunkt gleich weit vom Schnittpunkt entfernt liegen. Nun lässt sich das fehlende Stück mithilfe eines Kreisbogens schließen. Für kleine Geschwindigkeiten wird hierbei der Krümmungssprung zwischen Gerade und Kreisbogen toleriert. Die Berechnung des Kreisbogens erfolgt mit Konstruktion eines Gleichschenkligem Dreieck wie in Abbildung \ref{abb5} dargestellt.

\begin{figure}[H]
\flushleft
\begin{subfigure}{0.32\textwidth}
\includegraphics[width=0.95\textwidth]{fig/fig5a.tikz}
\end{subfigure}
\begin{subfigure}{0.32\textwidth}
    \includegraphics[width=0.95\textwidth]{fig/fig5b.tikz}
\end{subfigure}
\begin{subfigure}{0.32\textwidth}
    \includegraphics[width=0.95\textwidth]{fig/fig5c.tikz}
\end{subfigure}
\caption{Verbindung durch Gerade und Kreissegment}
\label{abb5}
\end{figure}

Der Radius des Kreises lässt sich nun über folgende Beziehung herleiten.

\begin{align}
\text{sin}(\frac{\alpha}{2}) &= \frac{\frac{d}{2}}{R} \\
&\Rightarrow \hspace{2cm} R = \frac{d}{2 \cdot \text{sin}(\frac{\alpha}{2})} \\
&\Rightarrow \hspace{2cm} l = R \cdot \alpha
\end{align}

Mit berechnetem Kreisradius und dem zu überbrückenden Winkel \(\alpha\), kann  ebenfalls die Länge des Kreissegments bestimmt werden. Somit ist der Kreis, welcher die fehlende Lücke schließt klar definiert und das Straßensegment geschlossen.

Für den Fall, dass die Straße an Start oder Endpunkt eine Krümmung besitzt, wird automatisch ein Spiralsegment eingefügt, welches den Krümmungssprung überbrückt. Ein weiterer Ausgleich ist erforderlich, falls die Anzahl der Spuren oder die Spurbreite zwischen Start und Endpunkt variiert.

Das Vorgehen gewährleistet einen kontinuierlichen Übergang vom Startpunkt in den Endpunkt und somit ein korrektes Schließen des Straßennetzwerkes. Zusammen mit den anderen gezeigten Funktionen bildet das Tool also eine Möglichkeit beliebige Straßennetze zu konstruieren und im OpenDRIVE Format abzuspeichern. Diese sind dann nutzbar für weitere Simulationen.

% --- BEISPIELE UND TESTS ------------------------------------------------------
\chapter{Beispiele und Tests}

Im folgenden Abschnitt finden sich einige beispielhaft generierte Kreuzungen. Zudem wurden im letzten Schritt einige zuvor generierte Straßen zusammengefügt und durch Verbindungsstraßen mithilfe der Funktion \texttt{closeRoadConnection} geschlossen.

\section{X-Kreuzung}
Wie schon erläutert gibt es für Kreuzungen verschiedene Typen. Der hier beispielhaft gezeigte Typ besteht aus einer durchgehenden Hauptstraße und zwei aneinandergrenzenden Nebenstraßen. Verbunden werden sollen alle Fahrstreifen im Kreuzungsbereich und die Kreuzungsbereichgröße wird auf 15 m in alle Richtungen gesetzt. Per Definition befindet sich der Kreuzungsmittelpunkt im Ursprung des x-y Koordinatensystems. Eine Darstellung der beispielhaften Kreuzung befindet sich in Abbildung \ref{abb6}.

\begin{figure}[H]
\flushleft
\includegraphics[width=0.95\textwidth]{fig/fig6.png}
\caption{X-Kreuzung im OpenDRIVE Viewer}
\label{abb6}
\end{figure}

Annahme innerhalb des Tools ist, dass Hauptstraßen eine zusätzliche Linksabbiegerspur besitzen, welche in der Abbildung sichtbar wird. Durch einen hier im Beispiel sehr großen Kreuzungsbereich werden die einzelnen Spuren auch ohne Fahrbahnmarkierung gut sichtbar. Automatisiert werden alle relevanten Spuren miteinander verbunden. Das Beispiel zeigt also eine valide Kreuzung generiert aus einfachen Eingabedaten, denn der Nutzer gibt neben den Gemometrien der drei Eingangsstraßen nur den Schnittwinkel an.

\section{Kreisverkehr}
Analog zur X-Kreuzung kann ein Kreisverkehrsegment erstellt werden. Ausgangspunkt ist die Definition des Kreises, der angrenzenden Straßen und jeweils die Position und der Winkel des Schnittpunkts. Daraus generiert das Tool alle Kreuzungsbereiche des Kreisverkehrs eigenständig, schneidet das Kreissegment an den entsprechenden Stellen und erstellt die Verbindungsstraßen der einzelnen Kreuzungen. Eine Visualisierung der Straßen lässt sich in Abbildung \ref{abb7} finden. Zudem werden automatisch Verkehrsschilder und Fahrbahnmarkierungen erstellt.

\begin{figure}[H]
\flushleft
\includegraphics[width=0.95\textwidth]{fig/fig7.png}
\caption{Kreisverkehr im OpenDRIVE Viewer}
\label{abb7}
\end{figure}

Angrenzend an eine ringförmige Straße treten vier Nebenstraßen auf, die zusammen mit Verbindungsstraßen einen geschlossenen, in sich validen Kreisverkehr ergeben.

\section{Komplexes Straßennetzwerk}

Diese und weitere parametrisierte Segmente lassen sich mithilfe des erstellten Tools generieren. Das zugrundeliegende Beispiel zeigt nun, dass die Segmente aneinander gefügt werden können und ferner zwei Straßenenden mit einer neu generierten Straße automatisiert miteinander Verbunden werden können. Zwar ist die erstellte Verbindungsstraße nicht immer die kürzeste Möglichkeit zur Verbindung der Straßen, jedoch wird der genaue Verlauf der Verbindungsstraße hierbei auch vernachlässigt.

\begin{figure}[H]
\flushleft
\includegraphics[width=0.95\textwidth]{fig/fig8.png}
\caption{Streckennetzwerk im OpenDRIVE Viewer}
\label{abb8}
\end{figure}

Zudem zeigt das Beispiel, welches in Abbildung \ref{abb8} zu finden ist weitere Funktionen des Tools, wie beispielsweise Spuraufweitungen oder einen Ausgleich von unterschiedlichen Spurbreiten. Zusätzlich sind beispielweise Verkehrsinseln, Parkplätze oder Bushaltestellen vor zu finden.

Insgesamt zeigen die dargestellten Beispiele, dass das Tool die vereinfachten Eingaben des Nutzers, zusammen mit Annahmen in ein valides und generisches Straßennetz im OpenDRIVE Standard übersetzen kann.

% --- AUSBLICK -----------------------------------------------------------------
\chapter{Ausblick}

Das in dieser Seminararbeit entwickelte Tool ermöglicht es mithilfe einer vereinfachten Streckenbeschreibung ein vollständiges Streckennetz im standardisierten OpenDRIVE Format zu generieren. Dies erlaubt Benutzern durch einfache Parameteränderungen komplett unterschiedliche, aber valide Strecken zu erhalten. So ist es möglich große Strecken mit wenigen Eingaben zu generieren und für eine anschließende Simulation zu benutzen.

Im Rahmen der Arbeit wurde der Streckengenerator in \texttt{C++} entwickelt und die Grundfunktionen implementiert. Selbstverständlich können aber weitere Annahmen getroffen werden und so eine noch intuitiver und einfachere Eingabe des Benutzers ermöglicht werden. Zudem können weitere Funktionen eingebaut werden, die dann weitere Szenarien abdecken. Beispiele hierfür sind Fußgängerüberwege oder Bürgersteige. Letztendlich ist durch das offen gehaltene OpenDRIVE Format jedes mögliche Szenario auch mithilfe des Generators in OpenDRIVE umsetzbar.

